%%%%%%%%%%%%%%%%%
% This is an sample CV template created using altacv.cls
% (v1.7.2, 28 August 2024) written by LianTze Lim (liantze@gmail.com). Compiles with pdfLaTeX, XeLaTeX and LuaLaTeX.
%
%% It may be distributed and/or modified under the
%% conditions of the LaTeX Project Public License, either version 1.3
%% of this license or (at your option) any later version.
%% The latest version of this license is in
%%    http://www.latex-project.org/lppl.txt
%% and version 1.3 or later is part of all distributions of LaTeX
%% version 2003/12/01 or later.
%%%%%%%%%%%%%%%%

%% Use the "normalphoto" option if you want a normal photo instead of cropped to a circle
% \documentclass[10pt,a4paper,normalphoto]{altacv}

\documentclass[10pt,letterpaper,ragged2e,withhyper]{altacv}
%% AltaCV uses the fontawesome5 and simpleicons packages.
%% See http://texdoc.net/pkg/fontawesome5 and http://texdoc.net/pkg/simpleicons for full list of symbols.

% Change the page layout if you need to
\geometry{left=1.25cm,right=1.25cm,top=1.5cm,bottom=1.5cm,columnsep=1.2cm}

% The paracol package lets you typeset columns of text in parallel
\usepackage{paracol}

% Change the font if you want to, depending on whether
% you're using pdflatex or xelatex/lualatex
% WHEN COMPILING WITH XELATEX PLEASE USE
% xelatex -shell-escape -output-driver="xdvipdfmx -z 0" sample.tex
\ifxetexorluatex
  % If using xelatex or lualatex:
  \setmainfont{Roboto Slab}
  \setsansfont{Lato}
  \renewcommand{\familydefault}{\sfdefault}
\else
  % If using pdflatex:
  \usepackage[rm]{roboto}
  \usepackage[defaultsans]{lato}
  % \usepackage{sourcesanspro}
  \renewcommand{\familydefault}{\sfdefault}
\fi

% Change the colours if you want to
%TODO choose and implement color scheme
\definecolor{SlateGrey}{HTML}{2E2E2E}
\definecolor{LightGrey}{HTML}{666666}
\definecolor{DarkPastelRed}{HTML}{450808}
\definecolor{PastelRed}{HTML}{8F0D0D}
\definecolor{GoldenEarth}{HTML}{E7D192}
\colorlet{name}{black}
\colorlet{tagline}{PastelRed}
\colorlet{heading}{DarkPastelRed}
\colorlet{headingrule}{GoldenEarth}
\colorlet{subheading}{PastelRed}
\colorlet{accent}{PastelRed}
\colorlet{emphasis}{SlateGrey}
\colorlet{body}{LightGrey}

% Change some fonts, if necessary
\renewcommand{\namefont}{\Huge\rmfamily\bfseries}
\renewcommand{\personalinfofont}{\footnotesize}
\renewcommand{\cvsectionfont}{\LARGE\rmfamily\bfseries}
\renewcommand{\cvsubsectionfont}{\large\bfseries}


% Change the bullets for itemize and rating marker
% for \cvskill if you want to
\renewcommand{\cvItemMarker}{{\small\textbullet}}
\renewcommand{\cvRatingMarker}{\faCircle}
% ...and the markers for the date/location for \cvevent
% \renewcommand{\cvDateMarker}{\faCalendar*[regular]}
% \renewcommand{\cvLocationMarker}{\faMapMarker*}


% If your CV/résumé is in a language other than English,
% then you probably want to change these so that when you
% copy-paste from the PDF or run pdftotext, the location
% and date marker icons for \cvevent will paste as correct
% translations. For example Spanish:
% \renewcommand{\locationname}{Ubicación}
% \renewcommand{\datename}{Fecha}


%% Use (and optionally edit if necessary) this .tex if you
%% want to use an author-year reference style like APA(6)
%% for your publication list
% \input{pubs-authoryear.cfg}

%% Use (and optionally edit if necessary) this .tex if you
%% want an originally numerical reference style like IEEE
%% for your publication list
%\input{pubs-num.cfg}

%% sample.bib contains your publications
%\addbibresource{sample.bib}

\begin{document}
\name{Alex Harter}
\tagline{IT Specialist}
%% You can add multiple photos on the left or right
\photoR{2.8cm}{Globe_High}
% \photoL{2.5cm}{Yacht_High,Suitcase_High}

\personalinfo{%
  % Not all of these are required!
  \email{alex@harter.tech}
  \phone{(940) 597-4771}
  % \mailaddress{Åddrésş, Street, 00000 Cóuntry}
  \location{Houston, TX}
  \homepage{https://alex.harter.tech}
  % \twitter{@twitterhandle}
  % \xtwitter{@x-handle}
  \linkedin{alexhartertech}
  \github{alexhartertech}
  % \orcid{0000-0000-0000-0000}
  %% You can add your own arbitrary detail with
  %% \printinfo{symbol}{detail}[optional hyperlink prefix]
  % \printinfo{\faPaw}{Hey ho!}[https://example.com/]

  %% Or you can declare your own field with
  %% \NewInfoFiled{fieldname}{symbol}[optional hyperlink prefix] and use it:
  % \NewInfoField{gitlab}{\faGitlab}[https://gitlab.com/]
  % \gitlab{your_id}
  %%
  %% For services and platforms like Mastodon where there isn't a
  %% straightforward relation between the user ID/nickname and the hyperlink,
  %% you can use \printinfo directly e.g.
  % \printinfo{\faMastodon}{@username@instace}[https://instance.url/@username]
  %% But if you absolutely want to create new dedicated info fields for
  %% such platforms, then use \NewInfoField* with a star:
  % \NewInfoField*{mastodon}{\faMastodon}
  %% then you can use \mastodon, with TWO arguments where the 2nd argument is
  %% the full hyperlink.
  % \mastodon{@username@instance}{https://instance.url/@username}
}

\makecvheader
%% Depending on your tastes, you may want to make fonts of itemize environments slightly smaller
% \AtBeginEnvironment{itemize}{\small}

%% Set the left/right column width ratio to 6:4.
\columnratio{0.6}

% Start a 2-column paracol. Both the left and right columns will automatically
% break across pages if things get too long.
\begin{paracol}{2}
\cvsection{Experience}

\cvevent{Musician}{Self-Employed}{November 2021 -- Ongoing}
\begin{itemize}
\item /Software/ - Utilize publishing and audio software to create and transmit sheet music, service booklets, and recordings.  Implemented morning service booklet which is now used every week.
\item /Performance & Leadership/ - Lead choirs in various services by singing, conducting, and giving directions in order to make services more orderly and encourage participation.  Plan and lead rehearsals.
\item /Mentoring & Collaboration/ - Suggest operational improvements and share resources with other choir directors.  Collaborate with experts in the field.  Working on adapting Latin and Slavonic chants into Spanish and English; producing a critical edition of Gregorian chant according to 1st millennium manuscripts.
\item /Education & Teaching/ - Educate myself and others about musical, linguistic, and historical aspects of Ukrainian and Latin church singing through primary source documents, academic papers, lectures, and communicating with experts.  Learned Cyrillic alphabet as well as old and new Gregorian and Kyivan notations.
\end{itemize}

\divider

\cvevent{Data Automation Consultant}{Encapture}{July 2022 -- February 2023}{Dallas, TX}
%TODO put Stephen's quote about my performance
\begin{itemize}
\item /Data Extraction/ - Extracted client-requested information from received documents sorted by classification algorithms then validated and translated that data according to customer specifications.
\item /Configuration of Pipeline/ - Configured internal software within Amazon Web Services to receive and sort customer documents and apply appropriate automation templates at multiple stages.
\item /Presentation/ - Demoed in-progress pipelines to clients to present progress, ask questions, and teach how to interface with the products.  Communicated internal software issues to the Product team with ideas for improvement for various clients.
\item /Documentation/ - Edited and contributed to documentation on Classification software for internal/client use.
\item /Team-Building/ - Started a book club with co-workers to improve professional and personal skills.
\end{itemize}

\divider

\cvevent{Choir Teacher / Special Education Paraprofessional / Afterschool Care Worker}{September 2019 -- November 2021}

\cvsection{Projects}

\cvevent{Project 1}{Funding agency/institution}{}{}
\begin{itemize}
\item Details
\end{itemize}

\divider

\cvevent{Project 2}{Funding agency/institution}{Project duration}{}
A short abstract would also work.

\medskip

\cvsection{A Day of My Life}

% Adapted from @Jake's answer from http://tex.stackexchange.com/a/82729/226
% \wheelchart{outer radius}{inner radius}{
% comma-separated list of value/text width/color/detail}
\wheelchart{1.5cm}{0.5cm}{%
  6/8em/accent!30/{Sleep,\\beautiful sleep},
  3/8em/accent!40/Hopeful novelist by night,
  8/8em/accent!60/Daytime job,
  2/10em/accent/Sports and relaxation,
  5/6em/accent!20/Spending time with family
}

% use ONLY \newpage if you want to force a page break for
% ONLY the current column
\newpage

\cvsection{Publications}

%% Specify your last name(s) and first name(s) as given in the .bib to automatically bold your own name in the publications list.
%% One caveat: You need to write \bibnamedelima where there's a space in your name for this to work properly; or write \bibnamedelimi if you use initials in the .bib
%% You can specify multiple names, especially if you have changed your name or if you need to highlight multiple authors.
\mynames{Lim/Lian\bibnamedelima Tze,
  Wong/Lian\bibnamedelima Tze,
  Lim/Tracy,
  Lim/L.\bibnamedelimi T.}
%% MAKE SURE THERE IS NO SPACE AFTER THE FINAL NAME IN YOUR \mynames LIST

\nocite{*}

\printbibliography[heading=pubtype,title={\printinfo{\faBook}{Books}},type=book]

\divider

\printbibliography[heading=pubtype,title={\printinfo{\faFile*[regular]}{Journal Articles}},type=article]

\divider

\printbibliography[heading=pubtype,title={\printinfo{\faUsers}{Conference Proceedings}},type=inproceedings]

%% Switch to the right column. This will now automatically move to the second
%% page if the content is too long.
\switchcolumn

\cvsection{My Life Philosophy}

\begin{quote}
``Something smart or heartfelt, preferably in one sentence.''
\end{quote}

\cvsection{Most Proud of}

\cvachievement{\faTrophy}{Fantastic Achievement}{and some details about it}

\divider

\cvachievement{\faHeartbeat}{Another achievement}{more details about it of course}

\divider

\cvachievement{\faHeartbeat}{Another achievement}{more details about it of course}

\cvsection{Strengths}

\cvtag{Hard-working}
\cvtag{Eye for detail}\\
\cvtag{Motivator \& Leader}

\divider\smallskip

\cvtag{Python}
\cvtag{Linux}\\
\cvtag{Statistical Analysis}
\cvtag{git}
\cvtag{SQL}
\cvtag{Regular Expressions}
\cvtag{Data Analysis}
\cvtag{Machine Learning}
\cvtag{Office Software}
\cvtag{Publishing Software}
\cvtag{Audio Software}


\cvsection{Languages}

\cvskill{English}{5}
\divider

\cvskill{Spanish}{2}
\divider

\cvskill{German}{1.5} %% Supports X.5 values.
\divider

\cvskill{Ukrainian}{1.5}

%% Yeah I didn't spend too much time making all the
%% spacing consistent... sorry. Use \smallskip, \medskip,
%% \bigskip, \vspace etc to make adjustments.
\medskip

\cvsection{Education}

\cvevent{CompTIA A+ (in progress)}{Sept 2002 -- June 2006}
Thesis title: Wonderful Research

\divider

\cvevent{Post-Graduate Certificate\ in Data Science and Business Analytics}{University of Texas at Austin}{August 2023 -- April 2024}

\divider

\cvevent{Content Certification\ in Music EC-12 & Special Education EC-12}{Texas Educator Certification Examination Program}

\divider

\cvevent{B.S.\ in Rehabilitation Studies}{University of North Texas}{Sept 2015 -- December 2019}

% \divider

\cvsection{References}

% \cvref{name}{email}{mailing address}
\cvref{Prof.\ Alpha Beta}{Institute}{a.beta@university.edu}
{Address Line 1\\Address line 2}

\divider

\cvref{Prof.\ Gamma Delta}{Institute}{g.delta@university.edu}
{Address Line 1\\Address line 2}


\end{paracol}


\end{document}
